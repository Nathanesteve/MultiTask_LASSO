\documentclass{article}
\usepackage{amsfonts}
\usepackage{amsmath}
\usepackage[utf8]{inputenc}
\begin{document}

%%%%%%%%%%%%%%%%%%%%%
\section*{De l'estimateur des moindres carrés au MultiTask LASSO} \vspace{.5cm}
%%%%%%%%%%%%%%%%%%%%%

\textbf{Méthode des moindres carrés ordinaire.} \\
\\ Le problème moindres carrés consiste à minimiser la fonction suivante: \\
$$\underset{\theta\in R^p}{\mathrm{argmin}} \ f(\theta)=\frac{1}{2} {\| y - X\theta\|_2^2} \
, \ y \in R^p,\ X \in \mathcal{M}(n,p).
 $$
 
La fonction $f$ étant convexe et différentiable, nous calculons le gradient de $f$ et l'annulons pour trouver son minimum global.
Le gradient de $f$ est donné par l'expression suivante\ :

\begin{align*}
    \nabla f(x) = \begin{pmatrix}
        \frac{\partial f}{\partial \theta_1} (\theta) \\ \vdots \\ \frac{\partial f}{\partial \theta_p} (\theta)
    \end{pmatrix}= \begin{pmatrix}
        x_1^T(X \theta -y) \\ \vdots \\ x_p^T(X \theta -y)
    \end{pmatrix} =X^T(X \theta -y)
\end{align*}

Soit $j \in \{1, ... ,p \}.$
	 
\begin{align*}
    \frac{\partial f}{\partial \theta_j} (\theta) = 0     &\Longleftrightarrow x_j^T(X \theta -y)= 0\\
     &\Longleftrightarrow x_j^T(x_j\theta_j + \sum_{k \ne j} x_k\theta_k - y) = 0\\ &\Longleftrightarrow \theta_j = \frac{x_j^T(y -  \sum_{k \ne j} x_k\theta_k)}{x_j^Tx_j} \\ &\Longleftrightarrow \theta_j = \frac{x_j^T(y -  \sum_{k=1}^{p} x_k\theta_k  + x_j\theta_j)}{\| x_j\|_2^2}
\end{align*} \\

La dernière écriture de $\theta_j$ va nous permettre, par un algorithme de descente par coordonnés, d'estimer le paramètre $\theta$ solution du problème de minimisation. \\

L'algorithme de descente par coordonnés permet de trouver une solution approché du problème: $\underset{\theta\in R^p}{\mathrm{argmin}} \ f(\theta).$  \\

Plus précisément, on transforme ce problème multidimensionnel en un problème unidimensionnel, en visitant les différentes coordonnées plusieurs fois et en les mettant à jour au fur et à mesure des itérations.\\
Ces visites régulières permettent d'assurer la convergence de l'algorithme vers la solution du problème.

\newpage 

\textbf{Régression LASSO.} \\

Le problème de la régression LASSO consiste à minimiser la fonction suivante: 

\begin{center}
$\underset{\theta\in R^p}{\mathrm{argmin}} \ f(\theta)=\frac{1}{2} {\| y - X\theta\|_2^2} + \ \lambda \sum_{k=1}^{p} |\theta_k|
, \ y \in R^p,X \in \mathcal{M}(n,p),\lambda > 0.
 $
\end{center}

La fonction $f$ est convexe mais non-différentiable car la fonction valeur absolue ne l'est pas.
Ainsi on ne pourra calculer le gradient de cette fonction et devrons alors traiter différent cas (en fonction du signe $t$ et $z$) pour déterminer, selon les cas, l'unique minimum de $f$.\\

Pour pouvoir utiliser un algorithme de descente par coordonnés, nous allons chercher a minimiser $f$ en $\theta_j$ avec $\theta_k$ fixé, $\forall\ k \ne j$.\\

Soit $j \in \{1, ... ,p \}$ et $ \hat{\theta_j} = \underset{\theta_j\in R}{\mathrm{argmin}} \ f(\theta).  $ 

\begin{align*}
\hat{\theta_j} &= \underset{\theta_j\in R}{\mathrm{argmin}}     \ \frac{1}{2} \|y - \sum_{k \ne j} x_k\theta_k  - x_j\theta_j\|^2 + \lambda \sum_{k \ne j}|\theta_k| + \lambda|\theta_j| \\ &= 
\frac{1}{2} \|x_j\|^2\theta_j^2 - <y - \sum_{k \ne j} x_k\theta_k,x_j>\theta_j + \lambda|\theta_j| \\ &=
\|x_j\|^2 \big[\frac{1}{2}(\theta_j - \|x_j\|^{-2}<y - \sum_{k \ne j} x_k\theta_k,x_j>)^2 + \frac{\lambda}{\|x_j\|^2}|\theta_j|]
\end{align*}


On considère $\eta_\lambda(z) = \underset{t \in R}{\mathrm{argmin}}\ \frac{1}{2} (z-t)^2 + \lambda|t| ,\ \lambda > 0\ et\ z \in R.\\ $ 

Nous avons donc: $\hat{\theta_j}=\eta_\frac{\lambda}{\|x_j\|^2}( \|x_j\|^{-2}<y - \sum_{k \ne j} x_k\theta_k,x_j>). \\ $

Par suite, il va nous falloir résoudre le problème de minimisation suivant: $$\underset{t \in R}{\mathrm{argmin}}\ \frac{1}{2} (z-t)^2 + \lambda|t| ,\ \lambda > 0\ et\ z \in R.\\ $$

Résolution du problème de minimisation.\\

Pour $z \in R$ , posons $f(t) = \frac{1}{2} (z-t)^2 + \lambda|t| \\ $ 

On considère 3 trois cas: \\
$\\$
$t < 0:\ f(t) = \frac{1}{2}(z-t)^2 + \lambda t \\ $ 
\newline
$t = 0:\ f(t) = \frac{1}{2}z^2 \\$
\newline
$t > 0:\ f(t) = \frac{1}{2}(z-t)^2 + \lambda t \\ $ 

Pour $t>0$ on a $f'(t) = t-z+\lambda$

$$\\ f'(t) = 0 \Longleftrightarrow t-z+ \lambda = 0\
\Longleftrightarrow t = z- \lambda.$$\
Or $t > 0$, ce qui signifie que $z- \lambda > 0$ et donc il est necessaire que $z > \lambda$.\\
De maniere analogue pour le cas $t < 0$, on obtient la condition $z < -\lambda.$ \\
Ainsi,\ on\ a:\\

$t = \left\{
    \begin{array}{lll}
        0 & \mbox{si } |z| \leq \lambda \\
        z - \lambda & \mbox{si } z > \lambda \\
        z+ \lambda & \mbox{si } z < - \lambda
    \end{array}
\right. \\ $ \\

On considère maintenant 5 cas en fonction de z:\\

Si $z > 0$ alors: 
$t = \left\{
    \begin{array}{lll}
         z - \lambda & \mbox{si } z > \lambda \\
         0 & \mbox{si }  z \leq \lambda 
    \end{array}
\right. \\ $
\vspace{0.5cm}

Si $z < 0$ alors: 
$t = \left\{
    \begin{array}{lll}
         z + \lambda & \mbox{si } z > - \lambda \\
         0 & \mbox{si }  -z \leq \lambda 
    \end{array}
\right.$
\vspace{0.5cm}

Si $z = 0$ alors $t = 0$.\\

En résumé:\\
 
$\eta(z) = \left\{
    \begin{array}{lll}
         0 & \mbox{si } |z| \leq \lambda \\
         z - \lambda & \mbox{si } z > \lambda \\
         z + \lambda & \mbox{si } z < - \lambda 
    \end{array}
\right. \\ $ \\

Ce qu'on peut réécrire de manière plus condensé comme suit:

$\eta(z) = max(|z| - \lambda,0)*sign(z)$
\\


Par conséquent:
 
\begin{align*}
\hat{\theta_j} &= \eta_\frac{\lambda}{\|x_j\|^2}( \|x_j\|^{-2}<y - \sum_{k \ne j} x_k\theta_k,x_j>) \\
&=  max(|\|x_j\|^{-2}<y - \sum_{k \ne j} x_k\theta_k,x_j>| -\frac{\lambda}{\|x_j\|^2},0)*sign(\|x_j\|^{-2}<y - \sum_{k \ne j} x_k\theta_k,x_j>)
\end{align*}










\end{document}

